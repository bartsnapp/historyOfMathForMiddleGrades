\newpage
\section{Friendly Pairs}

In this activity, we are going to play with pairs of numbers.  We'll
say that the pair $(a,b)$ is a ``friend'' to the pair $(c,d)$ if
$a\cdot d = b\cdot c$.

\begin{prob}
Give $5$ examples of pairs that are friends. Give $5$ examples of
pairs that are not friends.
\end{prob}

\begin{prob}\label{DF:2}
Is this a good rule?  That is, if $(a_1,b_1)$ is a friend to $(c,d)$
and $(a_2,b_2)$ is a friend to $(c,d)$, is it true that either
$(a_1,b_1)$ is a friend to $(a_2,b_2)$ or $(a_2,b_2)$ is a friend to
$(a_1,b_1)$?  If yes, justify your answer.  If no, give a
counterexample.
\end{prob}


Any counting number $a$ can be turned into a pair by the following
rule:
\[
a = (a,1)
\]

\begin{prob}\label{DF:3}
Suppose you can ``multiply'' pairs by the following rule:
\[
(a,b)\cdot (c,d) = (ac,bd)
\]
Is this a good rule, meaning do counting numbers still multiply the
correct way? Give three additional examples of pairs (that are not
integers) multiplying to make new pairs.
\end{prob}

\begin{prob}
Using the rules in Problem~\ref{DF:2} and Problem~\ref{DF:3} where now
the $a$, $b$, $c$, $d$, can themselves be pairs, simplify:
\[
\big((11,16),(3,5)\big)
\]
That is, show that $\big((11,16),(3,5)\big)$ is a friend to a simple
pair.
\end{prob}

\begin{prob}
What does any of this have to do with fractions? What does this have
to do with the ``invert and multiply'' rule for dividing fractions?
\end{prob}
