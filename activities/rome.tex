\newpage
\section{When in Rome\dots}

\begin{prob}
Count from 10--20 in the Hindu-Arabic, Roman, Mayan, Babylonian, and
Egyptian systems.
\end{prob}

\begin{prob}
Count from 100--200 \textit{by tens} in the Hindu-Arabic, Roman,
Mayan, Babylonian, and Egyptian systems.
\end{prob}

\begin{prob}
Count from 60--600 \textit{by sixties} in the Hindu-Arabic, Roman,
Mayan, Babylonian, and Egyptian systems.
\end{prob}

\begin{prob}
Discuss the advantages and disadvantages of the various systems. In
particular, can you distinguish between \textit{concatenation}
systems and \textit{place-value} systems?
\end{prob}

\begin{prob}
Adding and subtracting Roman numerals is perhaps easier than you may
think. For each of the problems above, give three examples of addition
problems using Roman numerals from the various ranges. See if you can
explain why Roman numerals were used in bookkeeping well into the
1700s. Big hint: Align like-letters when you can.
\end{prob}






