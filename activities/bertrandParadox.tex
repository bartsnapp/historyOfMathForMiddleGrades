\newpage
\section{Bertrand's Paradox}

In this activity we are going to investigate the following question:

\begin{quote}
Given a circle, find the probability that a chord chosen at random is longer than the side of an inscribed equilateral triangle. 
\end{quote}

\paragraph{Method 1: Random Endpoints}

\begin{prob}
We want to define our chord via ``random endpoints.'' Explain why,
without loss of generality, we may assume that one of the endpoints is
on the vertex of the triangle.
\end{prob}

\begin{prob}
Draw a picture of the situation and use the arcs of the circle defined
by the triangle to help you determine the probability in question.
\end{prob}




\paragraph{Method 2: Random Radius}

\begin{prob}
We want to define our chord via a random point on a ``random radius.''
Explain why, without loss of generality, we may assume that any chord
is a random point on some radius of the circle.
\end{prob}

\begin{prob}
Draw a picture of the situation and use the length of the radius (in
relationship to the inscribed triangle) to help you determine the
probability in question.
\end{prob}


\paragraph{Method 3: Random Midpoint}

\begin{prob}
We want to define our chord via its midpoint. Explain why a midpoint
will (almost!) always determine a unique chord of a circle. When does
it fail? 
\end{prob}

\begin{prob}
Give a compass and straightedge construction for a chord given its
midpoint.
\end{prob}

\begin{prob}
Draw a picture of the situation along with the incircle of the
equilateral triangle to help you determine the probability in
question.
\end{prob}



\paragraph{Conclusion?!}

\begin{prob}
What did you find? How do we resolve the paradox?
\end{prob}
