\newpage
\section{Owing Numbers}	

You are an Indian mathematician back in the year 650 working with
Brahmagupta and Bhaskara.  You know only the counting numbers and 0.
You learned in school how to add and multiply in this number system.

commutative. 

\begin{prob}
In school you learned:
\begin{itemize}
\item Both addition and multiplication were
associative and commutative. Explain what this means.
\item The number $0$ is an identity element for addition and 1 is an
identity element for multiplication. Explain what this means.
\item The distributive law holds.  Explain what this means.
\item The cancellation property holds, meaning:
\begin{align*}
a + b &= a + c  & a \cdot b &= a \cdot c\\
&\Rightarrow b = c & &\Rightarrow{b = c}
\end{align*}
Give examples of the cancellation properties. 
\end{itemize}
\end{prob}

\begin{prob}
Suppose there were some new numbers $e$ and $f$ such that
\[
e+ n = n\qquad\text{and}\qquad f\cdot n = n
\]
for every counting number $n$. Explain why $e$ must in fact be equal to
$0$ and $f$ must be in fact equal to $1$.
\end{prob}


\begin{prob}
Explain why given any counting number  $n$
\[
0 \cdot n = 0
\]
Big hint: Use the fact that $0 + 0 = 0$.
\end{prob}

You think the number system should be enlarged to
incorporate \textit{owing} an amount as well as \textit{having} an
amount. That's easy, just adjoin the numbers
$\{\tilde{1}, \tilde{2},\dots\}$.  Adding in this bigger number
system, that you call the integers, is easy, and you still have that
addition is associative and commutative, and that $0$ is still the
identity element for addition.  Defining multiplication of two numbers
$a\cdot b$ is not so easy when at least one of the two numbers is
negative.  You want to define it so that multiplication is still
associative and commutative, that $1$ is still the identity for
multiplication, and so that the distributive law continues to hold.


\begin{prob}
Explain why 
\[
m\cdot\tilde{n} = \tilde{m\cdot n}
\]
 where $m$ and $n$ are counting numbers (or $0$)?
\end{prob}


\begin{prob}
Use the problem above to explain why this forces you to define
\[
\tilde{m}\cdot \tilde{n} = m\cdot n 
\]
when $m$ and $n$ are counting numbers (or $0$)?
\end{prob}

\begin{prob}
What facts about negative numbers are we illustrating?
\end{prob}
