\newpage
\section{\sout{Deci}mals, \textit{Bi}-mals}


We use the decimal system to denote real numbers, but computers
don't. They use a ``sophisticated'' elaboration, \textit{bi-mals}.
That is, they would denote the number $101$ as:
\[
1\cdot 2^6 + 1\cdot 2^5 + 0 \cdot 2^4 + 0\cdot 2^3 + 1 \cdot 2^2 + 0 \cdot 2^1 + 1\cdot 2^0 = \text{``1100101''}
\]
Let's pretend like we are computers and work exclusively with bi-mals!

\begin{prob}
Calculate $1100+110$ by hand.
\end{prob}

\begin{prob}
Calculate $1100\cdot 110$ by hand.
\end{prob}


\begin{prob}
Calculate $1100101^{10}$ by hand. Remember that the exponent is a
bi-mal number too!
\end{prob}


Computers use the ``bi-mal point'' to write all real numbers in
bi-mal.  For example:
\[
1\cdot 2^2 + 1\cdot 2^1 + 0 \cdot 2^0 + 0\cdot 2^{-1} + 1 \cdot 2^{-2} + 0 \cdot 2^{-3} + 1\cdot 2^{-4} = \text{``110.0101''}
\]


\begin{prob}
Do the following bi-mal long division problem
\[
11\,\begin{array}[b]{@{}r@{}r} 
 & \\ 
\cline{1-1}
\big)\begin{array}[t]{@{}l@{}} 1100101 
\end{array}
\end{array}
\]
\end{prob}

\begin{prob} Express $\dfrac{1}{7}$ as a bi-mal. 
\end{prob}

\begin{prob} Which bi-mal fractions will terminate? Which will repeat?
\end{prob}

\begin{prob}
Calculate $110.0101^{10}$ by hand. 
\end{prob}


