\newpage
\section{Tenths are Best}

Again in the 17th and 18th centuries, the algebra of the number system
was being established.  Numbers were associated to lengths, areas,
weights, etc., so that putting things together corresponded to adding
their associated numbers.

\begin{prob}
Give three examples of assigning a number to measure the size or shape
or position of a geometric or physical object so that putting the
objects together corresponds to adding the numbers.
\end{prob}


\begin{prob}
Give three examples of assigning a number to measure the size or shape
or position of a geometric or physical object so that putting the
objects together does \textit{not} correspond to adding the numbers.
\end{prob}

\begin{prob}
One example of the first problem is to assign a number to a line
segment (called its length).  
\begin{enumerate}
\item I want my system to be such that, by telling the number to someone half-way round the world, they will be able to make a line segment of the same length, that is, a line segment congruent to mine.  What do I need to do to make such an
assignment?
\item If I want to have a number for every possible line
segment, what kind of a number system will need to have?  Is the
system of rational numbers enough? That is, can you always construct a
segment whose length is not a fraction?  Why or why not?
\end{enumerate}
\end{prob}

\begin{prob}
The previous discussion led us to the decimal number system.  Decimals can be finite or infinite.
\begin{enumerate}
\item Describe how to add two infinite decimals.
\item Describe how to multiply 
\[
(0.111111\dots)\cdot (0.333333\dots)
\]
\item Describe how to multiply any two (infinite) decimals.
\end{enumerate}
\end{prob}


\begin{prob}
What is the relationship between $0.999999\dots$ and $1$? Can you
explain this?
\end{prob}
