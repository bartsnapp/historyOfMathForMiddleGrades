\newpage
\section{Disk World}	


Once humans became comfortable with Euclidean Geometry, we took the
next step: Could the postulates and proofs be improved? In particular,
the Parallel Postulate seemed overly complicated. Could it
be \textit{proven} from the previous four, or was it really necessary?
It was not until the early 19th century that the mystery about the
role of the Parallel Postulate was solved and its absolute necessity
was shown. This was done by constructing geometries in which all of
Euclid's other postulates are true but the Parallel Postulate
is \textbf{false}.  Here is a construction that depends on two things:
\begin{enumerate}
\item Algebraic formulas about the distances between points on a circle. 
\item A geometric construction.
\end{enumerate}

Start with a circle of radius of $5$ cm, center $O$ and two points $A$
and $B$ inside it---place $A$ and $B$ near the edges.  The inside of
this circle will be called the \textbf{hyperbolic plane}.

\begin{prob} 
Use a ruler and calculator to calculate:
\[
\frac{5^2}{|OA|}.
\]
\end{prob}

\begin{prob} 
Extend the ray $\vec{OA}$ far enough outside the circle to be able to mark the point $A'$ such that
\[
|OA'| = \frac{5^2}{|OA|}.
\]
\end{prob}

\begin{prob}
Construct the circle passing through $A$, $A'$, and $B$.  Call the two
points where our two circles intersect $P$ and $Q$. 
\end{prob}

The points $P$ and $Q$ divide the newly constructed circle into two
arcs, the arc inside the hyperbolic plane and the arc outside the
hyperbolic plane.  Call the inside arc the \textbf{hyperbolic line}
passing through $A$ and $B$.

We need to have notions of \textit{distance} and \textit{angle} to
have a geometry.  The \textbf{angle} between two hyperbolic lines that
cross at a point $X$ is just the angle between the tangent lines to
the two pieces of circle. The \textbf{hyperbolic distance} between two
points, for example the two points $A$ and $B$ we started with, is
found via some complicated algebra:
\[
d_H(A,B) = \left|\ln\left(\frac{|AP|}{|BP|} \cdot \frac{|BQ|}{|AQ|} \right)\right|
\]

\begin{prob}
Does $d_H(A,B) = d_H(B,A)$?
\end{prob}

\begin{prob}
What happens to $d_H(A,B)$ when $A$ is fixed and $B$ gets closer and
closer to the edge of the circle?
\end{prob}

\begin{prob}
If $A$, $B$, and $C$ are on the same hyperbolic line, with $B$ between
$A$ and $C$, can you explain why $d_H(A,C) = d_H(A,B)+ d_H(B,C)$?
\end{prob}
