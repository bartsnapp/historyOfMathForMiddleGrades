\newpage
\section{A Whole New World}	

\begin{prob}
List Euclid's five postulates. Draw pictures representing these
postulates.
\end{prob}

\begin{prob}
Here are other statements closely related to Euclid's fifth postulate:
\begin{enumerate}
\item[(5A)] Exactly one line can be drawn through any point not on a given line parallel to the given line. 
\item[(5B)] The sum of the interior angles of every triangle is equal to $180^\circ$.
\item[(5C)] If two lines $\l_1$ and $\l_2$ are both perpendicular to some third line, then 
$\l_1$ and $\l_2$ do not meet.
\end{enumerate}
draw pictures depicting these statements.  Can you explain why
Euclid's fifth postulate is sometimes called the \emph{parallel
postulate}?
\end{prob}

There is a very natural geometry where the sum of the angles in every
triangle is \textit{greater than} $180^\circ$ and the essences of the
first four also still hold. Instead of working with a plane, we now
work on a sphere. We call this sort of geometry
\textit{Spherical Geometry}. Points, circles, angles, and
distances are exactly what we would expect them to be. But what do we
mean by lines on a sphere?  Lines are supposed to be extended
indefinitely.  In Spherical Geometry, the lines are the \emph{great
circles}. 

\begin{dfn}
A \textbf{great circle} is a circle on the sphere with the same center
as the sphere.
\end{dfn}

\begin{prob}
Draw some pictures of great circles and see if you can explain what
the definition above is saying.
\end{prob}

It is a theorem of Euclidean Geometry that the shortest path between
any two points on a plane is given by a line segment. We have a
similar theorem in Spherical Geometry.
\begin{thm}
The shortest path between any two points on a sphere is given by an
arc of a great circle.
\end{thm}


\begin{prob}
In Spherical Geometry, what is the difference between a great circle
and a regular Spherical Geometry circle?  Try to come up with a
definition of a \textit{circle} that will be true in both Euclidean
and Spherical Geometry.
\end{prob}

\begin{prob} Explain why the following proposition from Euclidean Geometry
  does not hold in Spherical Geometry: A triangle has at most one
  right angle.  (Can you find a triangle in Spherical Geometry with
  three right angles?)
\end{prob}

\begin{prob}
Explain why the following result from Euclidean Geometry does not hold
in Spherical Geometry: When the radius of a circle increases, its
circumference also increases.
\end{prob}

\begin{prob}
Define distinct lines to be \textbf{parallel} if they do not
intersect. Can you have parallel lines in Spherical Geometry?  Explain
why or why not.
\end{prob}


\begin{prob}
Come up with a definition of a \textit{polygon} that will be
true in both Euclidean and Spherical Geometry.
\end{prob}


\begin{prob}
A mathematician goes camping. She leaves her tent, walks one mile due
south, then one mile due east. She then sees a bear before walking one
mile north back to her tent. What color was the bear?
\end{prob}

\begin{prob}
The great German mathematician Gauss measured the angles of the
  triangle formed by the mountain peaks of Hohenhagen, Inselberg, and
  Brocken. What reasons might one have for doing this?
\end{prob}
