\newpage
\section{Voting Woes}	


In this activity, we are going to investigate the condrums that arise
when voting. With this in mind, I think it might be fitting to include
one of my favorite quotes:

\begin{quote}
If people do not believe that mathematics is simple, it is only because they do not realize how complicated life is.  

\hfill---John von Neumann
\end{quote}

With that said, let's go! There are lots of different voting
systems---here we describe three of them:

\paragraph{Plurality Voting:} Count the votes. The person with the most votes wins.
\paragraph{Vote For Two:} Everybody has two votes. Count the votes. The person with the most votes wins.
\paragraph{Borda Count:} Rank your candidates. Sum the ranks. The person with the \textbf{lowest} sum wins.



\begin{prob}
Cookie Monster, Big Bird, Grover
\end{prob}

\[
\begin{tabular}{|c|c|c|c|}\hline
Votes & First Preference & Second Preference & Third Preference & \\ \hline\hline
10    &  Cookie Monster  & 

\end{tabular}
\]

\begin{prob}

\end{prob}


Condorcet's Paradox
