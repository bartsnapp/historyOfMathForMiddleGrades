\newpage
\section{Estimating the Area of a Circle}


Draw a (fairly large) circle on a blank sheet of paper. We'll think of
this as a unit circle.


\begin{prob}
Divide the unit circle into $2^2 = 4$ equal wedges each with its vertex at
the center of the circle $O$.  On each wedge, call the two corners of
the wedge that lie on the circle $A$ and $B_2$.  Let $\mathcal{A}_2$
denote the area of the triangle $\tri OAB_2$ and let $\theta_2$ denote the
measure of the angle at $O$. Explain how to estimate the area of the
circle with triangle $\tri OAB_2$. What is your estimate?
\end{prob}

\begin{prob}
Divide the unit circle into $2^3 = 8$ equal wedges each with its vertex at
the center of the circle $O$.  On each wedge, call the two corners of
the wedge that lie on the circle $A$ and $B_3$.  Let $\mathcal{A}_3$
denote the area of the triangle $\tri OAB_3$ and let $\theta_3$
denote the measure of the angle at $O$. Explain how to estimate the
area of the circle with triangle $\tri OAB_3$. What information do
you need to know to actually do this computation?
\end{prob}

\begin{prob}
Given an angle $\theta$, explain the relation of $\sin(\theta)$ and
$\cos(\theta)$ to the unit circle. How could these values help with
the calculation described above?
\end{prob}

\begin{prob}
Divide the unit circle into $2^n$ equal wedges each with its vertex at
the center of the circle $O$.  On each wedge, call the two corners of
the wedge that lie on the circle $A$ and $B_n$.  Let $\mathcal{A}_n$
denote the area of the triangle $\tri OAB_n$ and let $\theta_n$
denote the measure of the angle at $O$. Explain why someone would be
interested in the value of:
\[
\sin\left(\frac{\theta_n}{2}\right)
\]
\end{prob}

\begin{prob}
Recalling that:
\[
\sin\left(\frac{\theta}{2}\right) = \sqrt{\frac{1-\cos(\theta)}{2}}
\qquad\text{and}\qquad
\cos(\theta)^2 + \sin(\theta)^2 = 1
\]
Explain why:
\[
2 \mathcal{A}_{n+1} = \sqrt{\frac{1 - \sqrt{1 - (2\mathcal{A}_n)^2}}{2}}
\]
\end{prob}


\begin{prob} 
Let's fill out the following table (a calculator will help!):
\[
\begin{array}{| c || c | c | c | c | c |}
\hline\bigstrut
n  & \mathcal{A}_n & \text{Approx. Area} & \sqrt{1-(2\mathcal{A}_n)^2} & \frac{1 - \sqrt{1-(2\mathcal{A}_n)^2} }{2} &\rule[0mm]{0mm}{6mm}2\mathcal{A}_{n+1} =\sqrt{\frac{1 - \sqrt{1-(2\mathcal{A}_n)^2} }{2}} \\ \hline\hline 
2 &\rule[7mm]{20mm}{0mm}\hspace{20mm}  &\hspace{20mm}  &\hspace{20mm}  & \hspace{20mm} & \hspace{20mm}\\ \hline
3 &\rule[0mm]{0mm}{7mm}   &  &  & &   \\ \hline
4 &\rule[0mm]{0mm}{7mm}   &  &  & &   \\ \hline
5 &\rule[0mm]{0mm}{7mm}   &  &  & &   \\ \hline
6 &\rule[0mm]{0mm}{7mm}   &  &  & &   \\ \hline
7 &\rule[0mm]{0mm}{7mm}   &  &  & &   \\ \hline
8 &\rule[0mm]{0mm}{7mm}   &  &  & &   \\ \hline
\end{array}
\]
What do you notice?
\end{prob}
