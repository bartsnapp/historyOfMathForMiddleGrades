\newpage
\section{Roll Play This}

Long before the \textit{World of Warcraft}, but not as long ago
as \textit{The Lord of the Rings}, there was a game
called \textit{Dungeons\&Dragons}. In this game you played a
``character'' with abilities such as strength, dexterity, and so
on. You computed these abilities by rolling a 6-sided die three times
and adding the numbers up (denoted 3d6). To give you a little feel for this, suppose
you were making your own character, and you were finding their
``Math-Class'' ability. Fill out the form below:
\[
\fbox{\begin{minipage}{60ex}
\vspace{.5cm}
Name:$\underset{\text{(make up a creative name for yourself)}}{\protect\rule[-2pt]{3in}{.5pt}}$\hfill\\
Math-Class Ability: $\underset{\text{(Roll 3d6)}}{\fbox{\rule[0mm]{0mm}{3ex}\hspace{3ex}}}$ 
\end{minipage}}
\]


Let's see how your character does in a made-up math class: (roll a
20-sided die 4 times and count how many times it rolls \textbf{below}
your ``Math-Class'' ability.)
\begin{itemize}
\item[4:] A congratulations!
\item[3:] B nice!
\item[2:] C no worries!
\item[1:] D is for diploma!
\item[0:] E umm\dots
\end{itemize}

As you can see, the game was thrilling. My question is this: Why do
they use the sum of rolling a 6-sided die 3 times to determine
the abilities? Let's see if we can figure this out.

\begin{prob}
Roll a 3d6 5 times and write down what you get. Demand that your
results are shared with the class and see what you get.
\end{prob}


\begin{prob}
Compute the \textit{relative frequency} of obtaining different results
for ``math ability.''
\end{prob}


\begin{prob}
Compute the \textit{probability} of obtaining different results
for ``math ability.''
\end{prob}

\begin{prob}
Can you explain why 3d6 was chosen to determine ability scores in the
game?
\end{prob}


