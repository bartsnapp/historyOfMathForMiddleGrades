\newpage
\section{Tic-Tac-D'oh}

In the 17th and 18th centuries, finally mathematics was coming into
its own.  The codification of algebra, the introduction of the complex
numbers, the discovery of the calculus, all these formed the analogy
in their day of the introduction of the personal computer and the
Internet in modern times.  There was, if you will, an explosion of
mathematics (although among a very small community---unlike the recent
technological revolution).

The sense was that mathematics was this huge new tool that could be
applied to understand many, many things, from the movement of the
planets to the workings of business and industry.  And it wasn't long
before it was applied to the ancient human activity of gambling.  The
key was knowing more than your opponent about how the game would come
out.  Let's do an experiment with the game tic-tac-toe:

\begin{prob}
Consider the following set-up:
\[
\textsf{\begin{tabular}{c|c|c}
o & & x \\ \hline
x & x & o\\ \hline
o &   & 
\end{tabular}}
\]
It is currently \textsf{x}'s turn. What is the probablity
that \textsf{x} wins assuming that the rest of the moves are random?
\end{prob}


\begin{prob}
Consider the following set-up:
\[
\textsf{\begin{tabular}{c|c|c}
  & &  \\ \hline
o & x & \\ \hline
 &   & 
\end{tabular}}
\]
It is currently \textsf{x}'s turn. What is the probablity
that \textsf{x} wins in the next three moves assuming that the rest of
the moves are random?
% three solns count boards/count games/prob
% 15/7C2*5C1  or 30/7*6*5 or 6/7 * 5/6 * 1/4
\end{prob}

