\documentclass[twoside,titlepage]{report}


\usepackage{amssymb,amsmath,amsfonts,amsthm,stmaryrd,fancyhdr,graphicx,color,multirow,pifont}
\usepackage[normalem]{ulem}
\usepackage{bigstrut}
\usepackage[all,cmtip]{xy}
\usepackage{microtype}

%% packages for the symbols activity:
%\usepackage{phaistos}
\usepackage{protosem}





%\usepackage{showkeys} %%% This package will show you your labels
                       %%% and should be commented out for the 
                       %%% final print out.
\usepackage{makeidx} %%% gives us our groovy index
%\usepackage[noDcommand]{kpfonts}

\makeindex

%\usepackage{layout} %%% Use command \layout in document to see margins
%%% Use package mathpazo to get another font

%%% These margins are set for 10 pt font size. 
%%% while some may find them to be too large, they 
%%% are set so that no more than 72 characters will
%%% be on any line.  This will help the readability 
%%% of the document.  If the font size is ever increased
%%% then the margins should be increased a bit too.
\oddsidemargin 62pt
\evensidemargin 62pt
\textwidth 345pt
\headheight 14pt




%%% Header and Footer Options
\renewcommand{\headrulewidth}{0.0pt} %%% this takes out the line 
%%%
%\rhead[]{\textsl{\leftmark}}  %%%
%\rfoot[]{\thepage}            %%% Options for 2 sided documents
%\lhead[\textsl{\rightmark}]{} %%%
%\lfoot[\thepage]{}            %%%
%%%
\lhead[\textsl{\rightmark}]{\textsl{\leftmark}} %%% Options for 1 sided documents
\cfoot[]{}                      %%% With typing on the *left* page
\rhead[]{}                                      %%% 
%%%
\cfoot[\thepage]{\thepage} %%% needed for both types of documents
%%% End Header and Footer Options


%%% This sets how the enumerate command works
\renewcommand{\theenumi}{$(\mathrm{\arabic{enumi}})$}
\renewcommand{\labelenumi}{\theenumi}
%\renewcommand{\topsep}{2em}
%%% This next bit of code defines all our theorem environments
\newtheoremstyle{SlantTheorem}{\topsep}{\topsep}%%% space between body and thm
		{\slshape}                      %%% Thm body font
		{}                              %%% Indent amount (empty = no indent)
		{\bfseries\sffamily}            %%% Thm head font
		{}                              %%% Punctuation after thm head
		{3ex}                           %%% Space after thm head
		{\thmname{#1}\thmnumber{ #2}\thmnote{ \bfseries(#3)}}%%% Thm head spec
\theoremstyle{SlantTheorem}
\newtheorem{thm}{Theorem}
\newtheorem*{para}{Paradox}
\newtheorem*{con}{Construction}
\newtheorem*{conj}{Conjecture}
\newtheorem{lem}[thm]{Lemma}
\newtheorem*{war}{WARNING}
\newtheorem*{eg}{Example}

\newtheoremstyle{problem}{\topsep}{\topsep}%%% space between body and thm
		{}                      %%% Thm body font
		{}                              %%% Indent amount (empty = no indent)
		{\bfseries}            %%% Thm head font
		{)}                    %%% Punctuation after thm head
		{ }                           %%% Space after thm head
		{\thmnumber{#2}\thmnote{ \bfseries (#3)}}%%% Thm head spec
\theoremstyle{problem}
\newtheorem{prob}{}[section]
\renewcommand{\theprob}{\arabic{prob}}

\newtheoremstyle{Definition}
{\topsep}{\topsep}{}{}{\sffamily\bfseries}{}{3ex}{}
\theoremstyle{Definition}
\newtheorem*{dfn}{Definition}

\newtheoremstyle{Exercises}
{\topsep}{\topsep}{}{}{\bfseries}{}{3ex}{}
\theoremstyle{Exercises}
\newtheorem*{ques}{Question}


\usepackage{array}
\setlength{\extrarowheight}{0cm}
\newdimen\digitwidth
\settowidth\digitwidth{9}
\def~{\hspace{\digitwidth}}
\def\divrule#1#2{
\noalign{\moveright#1\digitwidth
\vbox{\hrule width#2\digitwidth}}}



%%% This bit of code gives us our nice proof environment.
\renewenvironment{proof}[1][\proofname]{\begin{trivlist}\item[\hskip \labelsep \itshape \bfseries #1{}\hspace{2ex}]}
{\qed\end{trivlist}}
%%%

%%% This set of code gives us the unnumbered footnotes
\long\def\symbolfootnote[#1]#2{\begingroup\def\thefootnote{\fnsymbol{footnote}}
\footnote[#1]{#2}\endgroup}



%%% This set of code is all of our user defined commands
\newcommand{\bysame}{\mbox{\rule{3em}{.4pt}}\,}
\newcommand{\N}{\mathbb N}
\newcommand{\Z}{\mathbb Z}
\newcommand{\R}{\mathbb R}
\newcommand{\Q}{\mathbb Q}
\newcommand{\A}{\mathbb A}
\newcommand{\C}{\mathbb C}
\newcommand{\ph}{\varphi}
\newcommand{\ep}{\varepsilon}
\newcommand{\aph}{\alpha}
\newcommand{\QM}{\begin{center}{\huge\textbf{?}}\end{center}}
\renewcommand{\le}{\leqslant}
\renewcommand{\ge}{\geqslant}
\renewcommand{\a}{\wedge}
\renewcommand{\v}{\vee}
\renewcommand{\l}{\ell}
\renewcommand{\subset}{\subseteq}
\renewcommand{\supset}{\supseteq}
\renewcommand{\emptyset}{\varnothing}
\newcommand{\xto}{\xrightarrow}
\renewcommand{\qedsymbol}{$\blacksquare$}
\renewcommand{\bibname}{References and Further Reading}
\renewcommand{\abstractname}{Distributing this Document}
\renewcommand{\bar}{\protect\overline}
\renewcommand{\hat}{\protect\widehat}
\renewcommand{\tilde}{\widetilde}

\newcommand{\tri}{\triangle}

\newcommand{\mat}{\mathsf}
\renewcommand{\vec}{\overrightarrow}
\newcommand{\leftexp}[2]{{\vphantom{#2}}^{#1}{#2}}

\newcommand{\nocontentsline}[3]{}

\renewcommand*\thesection{\arabic{section}} % counts sections correctly


\begin{document}
\title{\textbf{\textsf{
\Huge History of Mathematics \\
\Large Math 2168: Spring 2013
}}}
\author{}
\date{}
\maketitle

%%%%
\cleardoublepage

\tableofcontents


\pagenumbering{arabic}
\pagestyle{fancy}
%%%%%%%%%%%%%%%%%%%%%%%%%%%%%%%%%%%%%%%%
%%%%%%% Sections to be included %%%%%%%%
%%%%%%%%%%%%%%%%%%%%%%%%%%%%%%%%%%%%%%%%

\renewcommand{\theenumi}{$(\mathrm{\alph{enumi}})$}
\renewcommand{\labelenumi}{\theenumi}

\newpage
\section{Food for Thought}	

Feel free to do these problems in any order.

\begin{prob}
What is a number? List out some qualities that you think a number has.
\end{prob}

\begin{prob}
How do you know when something is not a number?
\end{prob}

\begin{prob}
What are different ways we could represent numbers? 
\end{prob}

\begin{prob}
What if we used the following system:
\[
1 = a, \quad 2 = aa, \quad 3 = aaa, \quad\text{and so on.}
\]
What does the word \textit{concatenation} mean? How does it apply to
this system?  How would we add? How would we multiply? How could we
represent negative numbers?
\end{prob}


\begin{prob}
Once Oscar wondered what the number $\pi$ was. So he typed it into his
calculator and found:
\[
\pi = 3.1415926\dots
\]
Oscar then exclaimed, ``Ah now I know what number $\pi$ is.'' Can you
explain Oscar's thoughts on numbers?
\end{prob}


\input{../activities/rome.tex}

\input{../activities/pictureDivision.tex}

\input{../activities/friendlyFractions.tex}

\newpage
\section{Estimating the Area of a Circle}


Draw a (fairly large) circle on a blank sheet of paper. We'll think of
this as a unit circle.


\begin{prob}
Divide the unit circle into $2^2 = 4$ equal wedges each with its vertex at
the center of the circle $O$.  On each wedge, call the two corners of
the wedge that lie on the circle $A$ and $B_2$.  Let $\mathcal{A}_2$
denote the area of the triangle $\tri OAB_2$ and let $\theta_2$ denote the
measure of the angle at $O$. Explain how to estimate the area of the
circle with triangle $\tri OAB_2$. What is your estimate?
\end{prob}

\begin{prob}
Divide the unit circle into $2^3 = 8$ equal wedges each with its vertex at
the center of the circle $O$.  On each wedge, call the two corners of
the wedge that lie on the circle $A$ and $B_3$.  Let $\mathcal{A}_3$
denote the area of the triangle $\tri OAB_3$ and let $\theta_3$
denote the measure of the angle at $O$. Explain how to estimate the
area of the circle with triangle $\tri OAB_3$. What information do
you need to know to actually do this computation?
\end{prob}

\begin{prob}
Given an angle $\theta$, explain the relation of $\sin(\theta)$ and
$\cos(\theta)$ to the unit circle. How could these values help with
the calculation described above?
\end{prob}

\begin{prob}
Divide the unit circle into $2^n$ equal wedges each with its vertex at
the center of the circle $O$.  On each wedge, call the two corners of
the wedge that lie on the circle $A$ and $B_n$.  Let $\mathcal{A}_n$
denote the area of the triangle $\tri OAB_n$ and let $\theta_n$
denote the measure of the angle at $O$. Explain why someone would be
interested in the value of:
\[
\sin\left(\frac{\theta_n}{2}\right)
\]
\end{prob}

\begin{prob}
Recalling that:
\[
\sin\left(\frac{\theta}{2}\right) = \sqrt{\frac{1-\cos(\theta)}{2}}
\qquad\text{and}\qquad
\cos(\theta)^2 + \sin(\theta)^2 = 1
\]
Explain why:
\[
2 \mathcal{A}_{n+1} = \sqrt{\frac{1 - \sqrt{1 - (2\mathcal{A}_n)^2}}{2}}
\]
\end{prob}


\begin{prob} 
Let's fill out the following table (a calculator will help!):
\[
\begin{array}{| c || c | c | c | c | c |}
\hline\bigstrut
n  & \mathcal{A}_n & \text{Approx. Area} & \sqrt{1-(2\mathcal{A}_n)^2} & \frac{1 - \sqrt{1-(2\mathcal{A}_n)^2} }{2} &\rule[0mm]{0mm}{6mm}2\mathcal{A}_{n+1} =\sqrt{\frac{1 - \sqrt{1-(2\mathcal{A}_n)^2} }{2}} \\ \hline\hline 
2 &\rule[7mm]{20mm}{0mm}\hspace{20mm}  &\hspace{20mm}  &\hspace{20mm}  & \hspace{20mm} & \hspace{20mm}\\ \hline
3 &\rule[0mm]{0mm}{7mm}   &  &  & &   \\ \hline
4 &\rule[0mm]{0mm}{7mm}   &  &  & &   \\ \hline
5 &\rule[0mm]{0mm}{7mm}   &  &  & &   \\ \hline
6 &\rule[0mm]{0mm}{7mm}   &  &  & &   \\ \hline
7 &\rule[0mm]{0mm}{7mm}   &  &  & &   \\ \hline
8 &\rule[0mm]{0mm}{7mm}   &  &  & &   \\ \hline
\end{array}
\]
What do you notice?
\end{prob}


\input{../activities/euclid.tex}

\input{../activities/rightAngles.tex}

\input{../activities/cone.tex}

\input{../activities/tessPlatonic.tex}

\newpage
\section{Polyhedral Planets}	


In this activity, we will suppose in turn that you live on a planet
that is shaped like each of the regular convex polyhedra---not unlike
\textit{Le Petit Prince} who lives on an asteroid.

\begin{prob}
Considering each of the five regular convex polyhedra in turn, what
fraction of the planet's surface could you see if you stood:
\begin{enumerate}
\item In the middle of a face? 
\item In the middle of an edge?
\item On a vertex?
\end{enumerate}
In each case, explain your reasoning.
\end{prob}



\begin{prob}
Now suppose that you wish to go on a walk, surveying your polyhedral
planet. For each platonic solid, give the shortest path you can (draw
it on a net) that would allow you to observe the entire planet. In
each case, explain your reasoning.
\end{prob}


\input{../activities/symmetry.tex}

\input{../activities/negative.tex}

\input{../activities/trig.tex}

\newpage
\section{Geometry and Quadratic Equations}	



In ancient and Medieval times the discussion of quadratic equations
was often broken into three cases:
\begin{enumerate}
\item $x^2 + bx = c$
\item $x^2 = bx + c$
\item $x^2 + c = bx$
\end{enumerate}
where $b$ and $c$ are positive numbers. 


\begin{prob}
Create real-world word problems involving length and area for each
case above.
\end{prob}

\begin{prob}
Solve each of the three cases above by actually ``completing the
square'' using a \textbf{real} square.
\end{prob}

\begin{prob}
Is this a complete list of cases? If not, what is missing and why is
it (are they) missing? Explain your reasoning.
\end{prob}





\newpage
\section{State of the Art Circa 1550}	

Somewhere deep in your brain is a sleeping technique\dots AWAKE!
We want to solve:
\[
x^3 + 9x -26 =0
\]
We're going to have to use the Ferro-Tartaglia method, but all I can tell you are these three steps:
\begin{enumerate}
\item Replace $x$ with $u+v$. 
\item Set $uv$ so that all of the terms are eliminated except for $u^3$,
$v^3$, and constant terms.  
\item Clear denominators and use the quadratic formula.
\end{enumerate}


\begin{prob}
Use the Ferro-Tartaglia method to solve $x^3 + 9x -26 =0$.
\end{prob}


\begin{prob}
How many solutions should our equation above have? Where/what are they?
Hint: Make use of an old forgotten foe\dots
\end{prob}


\input{../activities/sphereGeo.tex}

\input{../activities/hyperbolicGeo.tex}

\input{../activities/projective.tex}

\input{../activities/locus.tex}

\newpage
\section{Complex Numbers From Different Angles}


In this activity we will investigate complex multiplication.

\begin{prob} Consider the innocent little equation:
\[
x^3 -1 = 0
\]
How many solutions does it have? What are they? Plot them on the
complex plane below.
\[
\includegraphics{../graphics/complexPlane.pdf}
\]
\end{prob}

\begin{prob}
Thinking about your work above, see if you can solve:
\[
x^4 -1 = 0
\]
Plot the solutions on the complex plane below.
\[
\includegraphics{../graphics/complexPlane.pdf}
\]
\end{prob}


\begin{prob}
Suppose I told you that:
\begin{align*}
\sin(x) &= x - \frac{x^3}{3!} + \frac{x^5}{5!} - \frac{x^7}{7!} + \dots + \frac{(-1)^n x^{2n+1}}{(2n+1)!} + \cdots \\
\cos(x) &= 1 - \frac{x^2}{2!} + \frac{x^4}{4!} - \frac{x^6}{6!} + \dots + \frac{(-1)^n x^{2n}}{(2n)!} + \cdots \\
e^x &= 1 + x + \frac{x^2}{2!} + \frac{x^3}{3!} + \frac{x^4}{4!} + \dots + \frac{x^n}{n!} + \cdots 
\end{align*}
Explain why we say:
\[
e^{x\cdot i} = \cos(x) + i \sin(x)
\]
\end{prob}

\begin{prob}
 This is Euler's famous formula:
\[
e^{\pi \cdot i } + 1 = 0
\]
Use the problem above to explain why it is true.
\end{prob}

\begin{prob}
What does all this have to do with De Moivre's Theorem?
\end{prob}

\begin{prob}
How can you use this to take the $n$th root of a complex number?
\end{prob}


\input{../activities/demoivre.tex}

%\newpage
\section{Attack of Ferro-Tartaglia}

You may recall that the Ferro-Tartaglia method gives 
\[
\sqrt[3]{\frac{-1+\sqrt{-31}}{2}} + \frac{2}{\sqrt[3]{\frac{1}{2}(-1+\sqrt{-31})}}
\]
as a solution to: $y = x^3-6x+1$. Moreover, as this plot of $y = x^3-6x+1$ shows
\[
\includegraphics[width=3in]{../graphics/cubicPlot.pdf}
\]
this must be a real solution\dots But how? There is a square-root of a
negative number in the expression above! Ok, uncharacteristically for
me, I'll show you how this is done. Please fasten your seat belts and
place your trays in the upright position.

Start by looking at:
\[
\frac{-1+\sqrt{-31}}{2} = \frac{-1}{2} + \frac{\sqrt{31}}{2} i 
\]
If we plot this in the complex plane, we get:
\[
\includegraphics[width=3in]{../graphics/CmplxPlane1.pdf}
\]
Using the distance formula, we see:
\[
\sqrt{\left( \frac{-1}{2} \right)^2 + \left(\frac{31}{2}\right)^2} = 2\sqrt{2}
\]
Hence we can plot the point $\left(\frac{-1}{4\sqrt{2}}, \frac{31}{4\sqrt{2}}\right)$ on the unit circle:
\[
\includegraphics[width=3in]{../graphics/CmplxPlane2.pdf}
\]
Since multiplication of complex numbers lying on the unit circle in the
complex plane is simply rotation, the cube-root of $\frac{-1}{2} +
\frac{31}{2} i$ is:
\[
\left(\sqrt[3]{2\sqrt{2}}\right)\cos\left(\arcsin\left(\frac{31}{4\sqrt{2}}\right)/3\right) + i \left(\sqrt[3]{2\sqrt{2}}\right) \sin\left( \arcsin\left(\frac{31}{4\sqrt{2}}\right)/3\right)
\]
Gosh, that's a mess, plotted on the complex plane, it looks like the black point below:
\[
\includegraphics[width=3in]{../graphics/CmplxPlane3.pdf}
\]

Now we'll compute the cube-root of: 
\begin{align*}
\frac{8}{\frac{1}{2}(-1+\sqrt{-31})} = \frac{-1}{2} + \frac{-\sqrt{31}}{2} i 
\end{align*}
If we plot this in the complex plane, we get:
\[
\includegraphics[width=3in]{../graphics/CmplxPlane4.pdf}
\]
Again, using the distance formula, we see:
\[
\sqrt{\left( \frac{-1}{2} \right)^2 + \left(\frac{31}{2}\right)^2} = 2\sqrt{2}
\]
Hence we can plot the point $\left(\frac{-1}{2}, \frac{-\sqrt{31}}{2} \right)$ on the unit circle:
\[
\includegraphics[width=3in]{../graphics/CmplxPlane5.pdf}
\]
Since multiplication of complex numbers lying on the unit circle in the
complex plane is simply rotation, the cube-root of $\frac{-1}{2} + \frac{-\sqrt{31}}{2} i $ is:
\[
\left(\sqrt[3]{2\sqrt{2}}\right)\cos\left(\arcsin\left(\frac{31}{4\sqrt{2}}\right)/3\right) - i \left(\sqrt[3]{2\sqrt{2}}\right) \sin\left( \arcsin\left(\frac{31}{4\sqrt{2}}\right)/3\right)
\]
Plotted on the complex plane, it looks like the black point below:
\[
\includegraphics[width=3in]{../graphics/CmplxPlane6.pdf}
\]
Putting all of this together, we find:
\[
\sqrt[3]{\frac{-1+\sqrt{-31}}{2}} + \frac{2}{\sqrt[3]{\frac{1}{2}(-1+\sqrt{-31})}} = \left(2\sqrt[3]{2\sqrt{2}}\right)\cos\left(\arcsin\left(\frac{31}{4\sqrt{2}}\right)/3\right)
\]
A messy, but real, number. 


\newpage
\section{Tic-Tac-D'oh}

In the 17th and 18th centuries, finally mathematics was coming into
its own.  The codification of algebra, the introduction of the complex
numbers, the discovery of the calculus, all these formed the analogy
in their day of the introduction of the personal computer and the
Internet in modern times.  There was, if you will, an explosion of
mathematics (although among a very small community---unlike the recent
technological revolution).

The sense was that mathematics was this huge new tool that could be
applied to understand many, many things, from the movement of the
planets to the workings of business and industry.  And it wasn't long
before it was applied to the ancient human activity of gambling.  The
key was knowing more than your opponent about how the game would come
out.  Let's do an experiment with the game tic-tac-toe:

\begin{prob}
Consider the following set-up:
\[
\textsf{\begin{tabular}{c|c|c}
o & & x \\ \hline
x & x & o\\ \hline
o &   & 
\end{tabular}}
\]
It is currently \textsf{x}'s turn. What is the probablity
that \textsf{x} wins assuming that the rest of the moves are random?
\end{prob}


\begin{prob}
Consider the following set-up:
\[
\textsf{\begin{tabular}{c|c|c}
  & &  \\ \hline
o & x & \\ \hline
 &   & 
\end{tabular}}
\]
It is currently \textsf{x}'s turn. What is the probablity
that \textsf{x} wins in the next three moves assuming that the rest of
the moves are random?
% three solns count boards/count games/prob
% 15/7C2*5C1  or 30/7*6*5 or 6/7 * 5/6 * 1/4
\end{prob}



\newpage
\section{Bertrand's Paradox}

In this activity we are going to investigate the following question:

\begin{quote}
Given a circle, find the probability that a chord chosen at random is longer than the side of an inscribed equilateral triangle. 
\end{quote}

\paragraph{Method 1: Random Endpoints}

\begin{prob}
We want to define our chord via ``random endpoints.'' Explain why,
without loss of generality, we may assume that one of the endpoints is
on the vertex of the triangle.
\end{prob}

\begin{prob}
Draw a picture of the situation and use the arcs of the circle defined
by the triangle to help you determine the probability in question.
\end{prob}




\paragraph{Method 2: Random Radius}

\begin{prob}
We want to define our chord via a random point on a ``random radius.''
Explain why, without loss of generality, we may assume that any chord
is a random point on some radius of the circle.
\end{prob}

\begin{prob}
Draw a picture of the situation and use the length of the radius (in
relationship to the inscribed triangle) to help you determine the
probability in question.
\end{prob}


\paragraph{Method 3: Random Midpoint}

\begin{prob}
We want to define our chord via its midpoint. Explain why a midpoint
will (almost!) always determine a unique chord of a circle. When does
it fail? 
\end{prob}

\begin{prob}
Give a compass and straightedge construction for a chord given its
midpoint.
\end{prob}

\begin{prob}
Draw a picture of the situation along with the incircle of the
equilateral triangle to help you determine the probability in
question.
\end{prob}



\paragraph{Conclusion?!}

\begin{prob}
What did you find? How do we resolve the paradox?
\end{prob}


\newpage
\section{Tenths are Best}

Again in the 17th and 18th centuries, the algebra of the number system
was being established.  Numbers were associated to lengths, areas,
weights, etc., so that putting things together corresponded to adding
their associated numbers.

\begin{prob}
Give three examples of assigning a number to measure the size or shape
or position of a geometric or physical object so that putting the
objects together corresponds to adding the numbers.
\end{prob}


\begin{prob}
Give three examples of assigning a number to measure the size or shape
or position of a geometric or physical object so that putting the
objects together does \textit{not} correspond to adding the numbers.
\end{prob}

\begin{prob}
One example of the first problem is to assign a number to a line
segment (called its length).  
\begin{enumerate}
\item I want my system to be such that, by telling the number to someone half-way round the world, they will be able to make a line segment of the same length, that is, a line segment congruent to mine.  What do I need to do to make such an
assignment?
\item If I want to have a number for every possible line
segment, what kind of a number system will need to have?  Is the
system of rational numbers enough? That is, can you always construct a
segment whose length is not a fraction?  Why or why not?
\end{enumerate}
\end{prob}

\begin{prob}
The previous discussion led us to the decimal number system.  Decimals can be finite or infinite.
\begin{enumerate}
\item Describe how to add two infinite decimals.
\item Describe how to multiply 
\[
(0.111111\dots)\cdot (0.333333\dots)
\]
\item Describe how to multiply any two (infinite) decimals.
\end{enumerate}
\end{prob}


\begin{prob}
What is the relationship between $0.999999\dots$ and $1$? Can you
explain this?
\end{prob}


%\newpage
\section{Aye, Me Parrot Concurs}


Ahoy ye mateys! Imagine you be a 17th century pirate, and you have
three systems o'measurement:
\begin{enumerate}
\item An \textit{ar} for length.
\item An \textit{arr} for width.
\item An \textit{arrr} for depth. 
\end{enumerate}
Howe'er, thin's be not quite as simple as they seem.

\begin{prob}
Aye, my sunken ship had a rectangular deck, with a perimeter o'34 ars.
Its hull be $12$ arrrs deep for a might storage hold o'180 ars.
\end{prob}


\begin{prob}
MORE
\end{prob}


Ahoy ye mateys! Imagine you be a 17th century pirate, and you have
three systems o'measurement:
\begin{enumerate}
\item An \textit{ar} for length.
\item An \textit{arr} for area.
\item An \textit{arrr} for volume. 
\end{enumerate}
Howe'er, thin's be not quite as simple as they seem.

\begin{prob}
Aye, my ship be a rectangular deck, with a perimeter o'30 ars.
But it
\end{prob}




\input{../activities/godzilla.tex}

\input{../activities/dioFermat.tex}

\input{../activities/cantor.tex}

\input{../activities/multipleChoice.tex}

\newpage
\section{The Applied Side---Stat!}

While Cantor was doing very abstract and pure mathematics in Germany,
statistics was making its first appearance as a mathematical subject
in England.  In this activity, we are going to investigate the \textit{mean}, \textit{median}, and \textit{mode} of a set of data. 


\begin{prob}
Explain what is meant by the mean, median, and mode of a set.
\end{prob}

\begin{prob}
Give a set of data consisting of $6$ values (not all equal!) such that
the mean, median, and mode are all equal. Plot your data in some
reasonable way.
\end{prob}

\begin{prob}
Give a set of data consisting of $6$ values such that the mean is
larger than both the median and mode which are equal. Plot your data in
some reasonable way.
\end{prob}


\begin{prob}
Give a set of data consisting of $6$ values such that the mode is
larger than both the mean and median. Plot your data in some
reasonable way.
\end{prob}

\paragraph{Simpson's Paradox}


In the 1970's a scandal arose at the University of California,
Berkley. There was a distinct gender bias against women in the number
of graduate students admitted among the 6 largest departments:
\[
\begin{tabular}{|c||c|c|} \hline
 & Applicants & Admitted \\ \hline\hline
Men & 2590 & 46\% \\ \hline
Women & 1835 & 30\% \\ \hline
\end{tabular}
\]
Data from individual departments was collected for further investigation:
\[
\begin{tabular}{|c||c|c||c|c|} \hline
\multirow{2}{*}{Department} & \multicolumn{2}{c||}{Men} & \multicolumn{2}{c|}{Women}  \\\cline{2-5}
 & Applicants & Admitted & Applicants & Admitted \\\hline\hline
A & 825 & 62\% & 108 & 82\%\\ \hline
B & 560 & 63\% & 25 & 68\% \\ \hline
C & 325 & 37\% & 593 & 34\% \\ \hline
D & 417 & 33\% & 375 & 35\% \\ \hline
E & 191 & 28\% & 393 & 24\% \\ \hline
F & 272 & 6\% & 341 & 7\% \\ \hline
\end{tabular}
\]

\begin{prob}
Examine this data critically. What seems to be the case? Does this
jive with your intuition? What is actually happening? Can you explain
when this will happen?
\end{prob}


%Perhaps it is not surprising that a cousin of Charles
%Darwin was involve, inventing correlation and regression as
%statistical methods to study the evolution of the human species.

%As an example of what correlation means, suppose there were four sets
%of twin females that were split up at birth with one twin living her
%life in the U.S. and the other twin living her life in England.
%Suppose the average adult height of a female in the U.S.\ is 5'4'' with
%a standard deviation of 2'' and the average adult height of a female in
%England is 5'6'' with a standard deviation of 3''.
%\[
%\begin{tabular}{|l||l|l|}\hline
%Twin set & U.S. & U.K. \\ \hline\hline
%1 & 5'6'' & 5'6'' \\\hline
%2 & 5'5'' & 5'4'' \\\hline
%3 & 5'6'' & 5'8'' \\\hline
%4 & 4'11'' & 5'6'' \\\hline
%\end{tabular}
%\]



%\begin{prob}
%Compute the correlation coefficient of this paired (twin) data using
%either formula for $r$ below.  

%Recall two methods of computing the \textit{sample correlation
%coefficient}, commonly denoted $r$:
%\[
%r = \frac{\sum_{i=1}^n(X_i-\bar{X})(Y_i-\bar{Y})}{\sqrt{\sum_{i=1}^n(X_i-\bar{X%})^2}\sqrt{\sum_{i=1}^n(Y_i-\bar{Y})^2}}
%\]
%An equivalent expression gives the correlation coefficient as the mean
%of the products of the standard scores.  Based on a sample of paired
%data $(X_i, Y_i)$, the sample correlation coefficient is:
%\[
%r = \frac{1}{n-1}\sum_{i=1}^n \left(\frac{X_i-\bar{X}}{\sigma_X} \right)\left(\frac{Y_i-\bar{Y}}{\sigma_Y} \right)
%\]
%\end{prob}

%\begin{prob}
%Figure out a single change in the data that makes the correlation
%coefficient bigger.
%\end{prob}



%Recall that Pearson's correlation coefficient between two variables is
%defined as the covariance of the two variables divided by the product
%of their standard deviations:
%\[
%\rho_{X,Y} = \frac{\mathrm{cov}(X,Y)}{\sigma_X\sigma_Y} = \frac{E[(X-\mu_X)(Y-\%mu_Y)]}{\sigma_X\sigma_Y}
%\]
%The above formula defines the \textit{population correlation
%coefficient}, commonly represented by the Greek letter $\rho$
%(rho). Substituting estimates of the covariances and variances based
%on a sample gives the
%\textit{sample correlation coefficient}, commonly denoted $r$:


%\newpage
\section{Voting Woes}	


In this activity, we are going to investigate the condrums that arise
when voting. With this in mind, I think it might be fitting to include
one of my favorite quotes:

\begin{quote}
If people do not believe that mathematics is simple, it is only because they do not realize how complicated life is.  

\hfill---John von Neumann
\end{quote}

With that said, let's go! There are lots of different voting
systems---here we describe three of them:

\paragraph{Plurality Voting:} Count the votes. The person with the most votes wins.
\paragraph{Vote For Two:} Everybody has two votes. Count the votes. The person with the most votes wins.
\paragraph{Borda Count:} Rank your candidates. Sum the ranks. The person with the \textbf{lowest} sum wins.



\begin{prob}
Cookie Monster, Big Bird, Grover
\end{prob}

\[
\begin{tabular}{|c|c|c|c|}\hline
Votes & First Preference & Second Preference & Third Preference & \\ \hline\hline
10    &  Cookie Monster  & 

\end{tabular}
\]

\begin{prob}

\end{prob}


Condorcet's Paradox



\newpage
\section{\sout{Deci}mals, \textit{Bi}-mals}


We use the decimal system to denote real numbers, but computers
don't. They use a ``sophisticated'' elaboration, \textit{bi-mals}.
That is, they would denote the number $101$ as:
\[
1\cdot 2^6 + 1\cdot 2^5 + 0 \cdot 2^4 + 0\cdot 2^3 + 1 \cdot 2^2 + 0 \cdot 2^1 + 1\cdot 2^0 = \text{``1100101''}
\]
Let's pretend like we are computers and work exclusively with bi-mals!

\begin{prob}
Calculate $1100+110$ by hand.
\end{prob}

\begin{prob}
Calculate $1100\cdot 110$ by hand.
\end{prob}


\begin{prob}
Calculate $1100101^{10}$ by hand. Remember that the exponent is a
bi-mal number too!
\end{prob}


Computers use the ``bi-mal point'' to write all real numbers in
bi-mal.  For example:
\[
1\cdot 2^2 + 1\cdot 2^1 + 0 \cdot 2^0 + 0\cdot 2^{-1} + 1 \cdot 2^{-2} + 0 \cdot 2^{-3} + 1\cdot 2^{-4} = \text{``110.0101''}
\]


\begin{prob}
Do the following bi-mal long division problem
\[
11\,\begin{array}[b]{@{}r@{}r} 
 & \\ 
\cline{1-1}
\big)\begin{array}[t]{@{}l@{}} 1100101 
\end{array}
\end{array}
\]
\end{prob}

\begin{prob} Express $\dfrac{1}{7}$ as a bi-mal. 
\end{prob}

\begin{prob} Which bi-mal fractions will terminate? Which will repeat?
\end{prob}

\begin{prob}
Calculate $110.0101^{10}$ by hand. 
\end{prob}




\input{../activities/dnd.tex}

\newpage
\section{It's How You Play The Game}

Let's play a game! Here is how it will work: Taking turns, each of you
will pick a square on the chart below to be ``yours.'' Continue until
every square is chosen.
\[
\begin{array}{|c|c|c|c|c|c|} \hline
1 & 2 & 3 & 4 & 5 & 6 \\ \hline
7 & 8 & 9 & 10 & 11 & 12 \\ \hline
\end{array}
\]
Then you will roll some dice. If your number comes up, you get a
point! First to $10$ points wins.


\begin{prob}
Play the game with a $12$-sided die. Does it matter which square you
chose? Explain your thoughts on this.
\end{prob}

\begin{prob}
Play the game with two $6$-sided dice, adding their values
together. Does it matter which square you chose? Explain your thoughts
on this.
\end{prob}


\begin{prob}
Play the game with three  $4$-sided dice, adding their values
together. Does it matter which square you chose? Explain your thoughts
on this.
\end{prob}


\begin{prob}
Compute the probability of rolling different squares with a 12-sided
die, two 6-sided dice, and three 4-sided dice. Make three charts to
help you out. What do you notice?
\end{prob}

Now let's change the rules of the game. In this game, you get the number of points of the square you are on. So if you choose ``square 3'' and you roll a 3, you get 3 points! First to $80$ wins!

\begin{prob}
Now which is the best square to choose if you are playing with:
\begin{itemize}
\item One 12-sided die?
\item Two 6-sided dice?
\item Three 4-sided dice?
\end{itemize}
\end{prob}

Note, if you enjoyed this game, check out \textit{Settlers of Catan}.




\end{document}
